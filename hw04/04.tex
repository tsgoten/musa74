\documentclass[10pt, twocolumn]{article}
\author{Tarang Srivastava}
\usepackage{amsmath, amssymb, amsthm, chngcntr}
\usepackage{graphicx}
\usepackage[margin=.25in]{geometry}
\setlength{\columnsep}{.5in}
\newcommand{\makechaptertitle}[1]{
\begin{center}
	\begin{large}
		#1
	\end{large}
	\begin{small}
		\\Tarang Srivastava
	\end{small}
\end{center}
}
\theoremstyle{definition} 
\newtheorem{q}{}
\renewcommand*{\theq}{\alph{q}}
\counterwithin*{q}{section}
\begin{document}
	
\makechaptertitle{MUSA 74 Homework 3}

\begin{q}
Homework 1.67 \\
Proof. Proceed by induction on $ n $. For the base case consider when $ n = 1 $. That is, $ p|a_1 $
The base case is trvially true, and the statement holds.
Assume the statement holds for some $ k \in \mathbb{N} $.
That is, if $ p|a_i...a_k $, then there is some $ i \in \{1, ..., k\} $ such that $ p|a_i $.
In the inductive step, we need to show the statement holds for $ k + 1 $.
That is, if $ p|a_i...a_{k+1} $, then there is some $ i \in \{1, ..., k+1\} $ such that $ p|a_i $.
We will prove the inductive step by cases.
The first case is when $ i $ is in some set $ \{1, ..., k\} $.
From our inductive hypothesis we know there exists an $ i \in \{1, ..., k\} $ 
such that $ p|a_i $. 
The second case is when $ i $ is not in some set $ \{1, ..., k\} $.
Therefore, for all $ i \in \{1, ..., k\} $, $ p \not | a_i $. 
Which is equivalent to the $ \text{gcd}(p, a_i) = 1 $ for all $ i \in \{1, ..., k\} $, 
but since $ p|a_i...a_{k+1} $, and the gcd for $ p $ and $ a_i...a_{k+1} $ is $ p $. 
There must exist an $ a_i $ for $ i \in \{1,...,k+1\} $ such that gcd for $ p $ and $ a_i $ is $ p $ and therefore $ p|a_i $. 
Thus, having shown the inductive step the statement holds for all $ n $.
\end{q}
\begin{q}
Homework 1.68 \\
Proof. Since we are concerned about a Cartesian product an arbitrary $ n $ times we can proceed by induction on $ n $. 
For the base case of $ n = 0 $, $ \mathbb{N} $ is trivially countable.
Assume that for some Cartesian product $ k \in \mathbb{N} $ times, the product is countable.
We will show that the Cartesian product for $ k + 1 $ times is also countable. 
Let $ g: X_1 \times X_2 \times ... \times X_k \rightarrow \mathbb{X} $ from the inductive hypothesis.
Then we can define a bijective function $ f $ such that for $ f(2k) = g(2k) $ and $ f(2k+1) = n \in X $.
This defintion of $ f $ creates a bijection for the $ k + 1 $ case and therefore by induction the statement holds for all $ n $. \\
\end{q}
\begin{q}
Homework 1.69 \\
Proof. Proceed by induction on $ n $.
For the base case consider when $ n = 1$. 
The statement holds since, 
$$ 1^2 = \dfrac{1(2)(3)}{6} = 1 $$
For the inductice step assume the statement holds for some $ k $.
That is,
$$ 1^2 + 2^2 + ... + k^2 = \dfrac{k(k+1)(2k+1)}{6} $$
We will show the statement holds for $ k + 1 $. That is,
$$ 1^2 + 2^2 + ... + (k+1)^2 = \dfrac{(k+1)(k+2)(2k+3)}{6} $$
From our inductive hypothesis we can substitute in 
$$  \dfrac{k(k+1)(2k+1)}{6} + (k+1)^2 = \dfrac{(k+1)(k+2)(2k+3)}{6} $$
Which with some algebra is equivalent.

\end{q}

\end{document}